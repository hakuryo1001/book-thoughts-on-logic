
$$\rightarrow : \text{}$$
$$\leftrightarrow : \text{\ruby{}{嘅就兼如果}}$$
$$\iff : \text{係}$$
$$\vdash : \text{講得通} \sim \text{推到} \sim \text{證明到}  $$
$$\models : \text{意味} \sim \text{撐住}$$

\begin{table}[htbp]
% \begin{table}[H]
\centering


\begin{tabular}{|c|p{2cm}|p{5cm}|}
\hline
\textbf{Symbol} & \textbf{Cantonese} & \textbf{Example} \\
\hline
$\wedge$ & 同 & A 同 B 係堅,係 A 係堅 同埋 B 係堅。 \\
\hline
$\vee$ & 或者 & A 或者 B 係堅,係 A 係堅 或者 B 係堅。 \\
\hline
$\neg$ & 唔係 & 唔係 A 係堅,係 A 係流。 \\
\hline
$\rightarrow$ & 嘅就 & A 嘅就 B 係堅,係 A 係流 或者 B 係堅。 \\
\hline
$\leftrightarrow$ & 嘅就兼如果 & A 嘅就兼如果 B 係堅,係 A 堅 嘅就 B 堅 同 B 堅 嘅就 A 堅。 \\
\hline
$\oplus$ & 定 & A 定 B 係堅,係 A 或者 B 其中一個係堅,但唔可以兩個都係堅。 \\
\hline
$\uparrow$ & 撞 & A 撞 B 係流,係 A 同 B 都係堅;如果唔係,就係堅。 \\
\hline
$\downarrow$ & 唔係...又唔係 & A 唔係兼唔係 B 係堅,係 A 同 B 都係流。 \\
\hline



$\leftrightarrow$& 嘅就兼如果& \\
\hline
$\iff$ & 姐係 &\\
\hline
$\vdash$ & 嘅就講得通 \newline 嘅就推到 \newline  嘅就證明到  & \\
\hline
$\models$ & 嘅就意味 \newline  嘅就撐住 & \\
\hline
\end{tabular}
\caption{Cantonese Logical Constructions}
\end{table}
