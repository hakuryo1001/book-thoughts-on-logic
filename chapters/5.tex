

1. **Correspondence Theory**
    真相就係同現實一樣嘅嘢。
    (*Truth is what matches what really happens.*)

2. **Coherence Theory**
    真相就係同你信開嗰套講得埋一齊嘅嘢。
    (*Truth is what fits with the rest of what you believe.*)

3. **Pragmatic Theory**
    真相就係實際上行得通、有用嘅嘢。
    (*Truth is what actually works or helps in real life.*)

4. **Deflationary Theory**
    真相就係講嚟方便啫,冇乜特別意思。
    (*Truth is just a way of talking — nothing deep behind it.*)

5. **Constructivist Theory**
    真相就係大家一齊傾、一齊揼出嚟嘅嘢。
    (*Truth is something people build together through talking and doing.*)

6. **Semantic Theory of Truth**
    真相就係講嘅嗰句同現實牙嘅嘢講。
    (*Truth is when what you say matches what’s real.*)

7. **Redundancy Theory**
    真相之不過係重複噏返你講過嘅嘢,加多句「係真架」之嘛,其實冇加多任何新意思。
    (*Truth is just repeating yourself — saying “it’s true” adds nothing new.*)

8. **Reliabilist Theory**
    真相就係用可靠方法搵到出嚟嘅嘢。
    (*Truth is what comes from a trustworthy way of finding things out.*)

9. **Pluralist Theory**
    真相就係分情況,有時咁、有時咁。
    (*Truth depends on the situation — different things can be true in different places.*)

10. **Alethic Realism**
     真相就係無論你信唔信,都照樣存在嘅嘢。
     (*Truth is something that’s there no matter what we think.*)
---
Here’s each “beauty is…” sentence rewritten in **natural, colloquial Cantonese**, all starting with **「乜嘢係靚就係……」**, using everyday Hong Kong tone (clear, friendly, no philosophical jargon):

---

1. **Classical Theory**
    乜嘢係靚就係有和諧、有比例、睇落舒服嘅嘢。
    (*Beauty is what looks balanced and harmonious.*)

2. **Romantic Theory**
    乜嘢係靚就係會令人心入面起感覺、好感動嘅嘢。
    (*Beauty is what touches your heart or stirs deep feelings.*)

3. **Aesthetic Experience Theory**
    乜嘢係靚就係你睇、聽、感受到之後覺得好享受嘅經驗。
    (*Beauty is what comes from an experience that feels rich and enjoyable.*)

4. **Subjective Theory**
    乜嘢係靚就係自己睇得開心、覺得正嘅嘢。
    (*Beauty is whatever makes you feel good or looks nice to you personally.*)

5. **Objective Theory**
    乜嘢係靚就係件物本身有啲特質會令人讚嘅嘢。
    (*Beauty is something that’s genuinely admirable because of what it is.*)

6. **Formalism**
    乜嘢係靚就係形狀、線條、結構整得靚嘅嘢。
    (*Beauty is found in the shapes, lines, and structure — how it’s made.*)

7. **Cultural Relativism**
    乜嘢係靚就係喺嗰個文化入面覺得靚嘅嘢。
    (*Beauty is whatever people in that culture think is beautiful.*)

8. **Evolutionary Theory**
    乜嘢係靚就係睇落健康、有活力、有吸引力嘅嘢。
    (*Beauty is what looks healthy, energetic, and attractive.*)

9. **Postmodern Theory**
    乜嘢係靚就係敢挑戰舊規矩、每個人睇法都唔同嘅嘢。
    (*Beauty is what breaks the rules and means different things to everyone.*)

10. **Utilitarian Theory**
     乜嘢係靚就係會令人開心、減少唔舒服嘅嘢。
     (*Beauty is what brings pleasure and makes life feel better.*)



    
Here are ten theories of the "good" along with their representative philosophers and summaries:

1. **Utilitarianism**  
   - **Philosophers**: Jeremy Bentham, John Stuart Mill  
   - **Good is what maximizes pleasure and minimizes pain.**

2. **Deontological Ethics**  
   - **Philosophers**: Immanuel Kant  
   - **Good is what aligns with moral duties and universal laws.**

3. **Virtue Ethics**  
   - **Philosophers**: Aristotle, Alasdair MacIntyre  
   - **Good is what promotes human flourishing and virtuous character.**

4. **Ethical Relativism**  
   - **Philosophers**: Ruth Benedict, Franz Boas  
   - **Good is what is defined by cultural norms and societal context.**

5. **Divine Command Theory**  
   - **Philosophers**: St. Augustine, William of Ockham  
   - **Good is what is commanded by a divine being.**

6. **Natural Law Theory**  
   - **Philosophers**: Thomas Aquinas, John Finnis  
   - **Good is what is in accordance with human nature and reason.**

7. **Hedonism**  
   - **Philosophers**: Epicurus, Jeremy Bentham  
   - **Good is what brings the greatest pleasure to the individual.**

8. **Social Contract Theory**  
   - **Philosophers**: Thomas Hobbes, John Locke, Jean-Jacques Rousseau  
   - **Good is what is agreed upon for the benefit of society.**

9. **Pragmatic Ethics**  
   - **Philosophers**: John Dewey, William James  
   - **Good is what proves effective in solving problems and improving lives.**

10. **Care Ethics**  
    - **Philosophers**: Carol Gilligan, Nel Noddings  
    - **Good is what nurtures relationships and promotes care for others.**


Here are ten theories of "the right" along with their representative philosophers and summaries:

1. **Utilitarianism**  
   - **Philosophers**: Jeremy Bentham, John Stuart Mill  
   - **The right is what maximizes pleasure and minimizes pain for the greatest number.**

2. **Deontological Ethics**  
   - **Philosophers**: Immanuel Kant  
   - **The right is what conforms to moral duties and categorical imperatives.**

3. **Virtue Ethics**  
   - **Philosophers**: Aristotle, Alasdair MacIntyre  
   - **The right is what cultivates virtuous character and promotes human flourishing.**

4. **Social Contract Theory**  
   - **Philosophers**: Thomas Hobbes, John Locke, Jean-Jacques Rousseau  
   - **The right is what is agreed upon for the benefit of society and its members.**

5. **Divine Command Theory**  
   - **Philosophers**: St. Augustine, William of Ockham  
   - **The right is what is commanded by God or aligns with divine will.**

6. **Consequentialism**  
   - **Philosophers**: Peter Singer, R. M. Hare  
   - **The right is determined by the outcomes and consequences of actions.**

7. **Rights Theory**  
   - **Philosophers**: John Locke, Robert Nozick  
   - **The right is what respects and upholds individual rights and freedoms.**

8. **Care Ethics**  
   - **Philosophers**: Carol Gilligan, Nel Noddings  
   - **The right is what fosters care and responsibility in relationships.**

9. **Pragmatic Ethics**  
   - **Philosophers**: John Dewey, William James  
   - **The right is what effectively addresses problems and promotes well-being.**

10. **Moral Intuitionism**  
    - **Philosophers**: G. E. Moore, W. D. Ross  
    - **The right is what is intuitively recognized as morally correct.**

