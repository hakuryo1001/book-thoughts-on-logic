\chapter{Reason and Rationality}
Is reason sufficient to morality? In my youth of course the answer is yes. But now I think the answer is a definite no. Reason alone cannot introduce axioms. MetaReason suffers the same problem in that we don’t know what motivates the axioms that build the metareason. Turtles all the way down. Sacredness (or something of that sort) is necessary. When Peter Singer asks why would it be moral to eat an pig but not a baby human despite they share a lot of similarities, or that a pig shares a lot of similaires with some subset of human babies, what one should read is, why can’t we eat human babies? Arbitrary lines unresolvable by reason can only resolved by something extrarational. 



If the axiom of choice being independent of the rest of the theoretical axioms implies that we are able to choose whether we accept axing of choice. You suggest to us that we can design or axiomatic system is to my liking and playing with a resultant object as a mathematical playthings. Again, a similar situation happens when we consider Euclidean axioms in the parallel postulate. Ian hacking says that because there are wagons and we can talk about wagons does not mean that we can. We need to assume the existence of wagonhood. He then goes on to argue with us just because we use numbers and we can describe a lot about numbers, As so many mathematicians do with numbers enthusiasm. It does not mean that we need to assume that they’re numbers. However, while we talk about neighbours we also assume the existence of neighbourhoods. The “hood” as a suffix merely postulated its existence thereof. We can choose whether we believe neighbourhoods wagonhoods and numbers exist, just like we can choose whether we accept the axiom of choice - or whether the action of choice is an endorsed action of creation. 


Why is circularity taken to be less acceptable than axioms accepted on basis of faith? 

Why can’t that faith be transferred to a circular justification?

Shouldn’t there be some romantic reason to believe, that if a justification, a justification for all other justifications, is its own justification, is the best justification of all? Shouldn’t that then give us the greatest reason to place that faith onto it? 

If a justification is based on another justification, and that other justification is based on the former justification - then this is circularity by definition. But it also means, those two justifications are equivalent to each other. 