\chapter{}

% Define tonemark command if not already defined
\providecommand{\tonemark}[2]{#1#2}
我  缺乏 講到 邏輯 嘅 時文性嘅 架罉。冚巴冷乜都冇,攪到開口閤著脷,半步出門就當災。叫係咁\tone{意}{2}開到頭嘅話就要返啲架罉。嗱,架罉呢點樣嚟呢?一就就好似啲古希臘人咁指接喺廣東話裏面提鍊出嚟,唔係就翻譯來佬野。前者就原汁原味,但係會做得好慢好辛苦,同埋如果要原生味道保留得到要基本上嗰做奠定功勞嘅條友要室方面功夫高,但又同時可以自我隔離到啲泰西邏輯嘅影響。後者呢,就快靚正——如果你唔會有啲咩外來物種鵲巢鳩佔我地隻話裏面嘅原生邏輯嘅話。但仲有一樣野——做無恥進口嘅話,廣東話嘅語文習慣好容易就會捩手不問三四直接攞普通話現成嘅嘢嚟用——咁嘅話做嚟姐多餘,嘥氣。

我地當然希望可以有機會可以做到原汁原味嘅提鍊啦,但係退而求其次嘅話,我地應該做嘅就係將泰西已經奠基咗嘅基礎邏輯連詞,喺我地廣東話裏面搵最類似嘅嘢,嚟本土貨當樣貨用。普通話嘅嘢一嘅唔用,盞危險。

按照住呢個邏輯呢,我地最徹底嘅最核爆式嘅做法,就係一句嘢啱唔啱到唔用「真」同「假」都唔用。我地要同自己講,同冚世界講:對我地嚟講,嘢講唔知乜嘢係真乜嘢係假。我地淨係知道乜嘢係「堅」,乜嘢係「流」。

其實都唔係咁離譜嘅啫,事關一句嘢講係咪「真」同「假」根本就同我地語文直覺答錯先唔埋\tone{冷}{1}——我都唔知乜嘢叫一句嘢講係真,一句嘢講係假。


咁樣,我地可以開始勒。


% A and B is true if and only if both A is true and B is true.

% A or B is true if at least one of A or B is true.

% Not A is true if and only if A is false.

% A implies B is false if and only if A is true and B is false; otherwise, it is true.

% A if and only if B is true if and only if A and B have the same truth value.

% A exclusive or B is true if and only if exactly one of A or B is true, but not both.

% A nand B is false if and only if both A and B are true; otherwise, it is true.

% A nor B is true if and only if neither A nor B is true.




% ---

% % ### Cantonese Logical Connective Equivalents

% % - AND (∧): 同 (tung4)
% % - OR (∨): 或者 (waak6 ze2)
% % - NOT (¬): 唔係 (m4 hai6)
% % - IMPLIES (→): 如果...咁 (jyu4 gwo2... gam2)
% % - IF AND ONLY IF (↔): 夾 (gaap3) (colloquial for things being identical/matching)
% % - XOR (⊕): 定 (ding6) (used to contrast two choices)
% % - NAND (↑): 撞 (zong6) (used to indicate conflict or unexpected combination)
% % - NOR (↓): 乜 (mat1) (negative intensifier, common in colloquial speech)
% % - Nor etymology
    
% % %     ## 1. "Nor" - A Fusion of Negatives
    
% % %     ### Etymology & Evolution
    
% % %     - Old English: *ne… ne* (e.g., *ne he ne ic* → "neither he nor I")
% % %     - Middle English: *ne… nor*
% % %     - Modern English: *nor*
    
% % %     The word "nor" comes from a combination of ne (not) + or, with ne being the core negation word in Old English.
    
% % %     ### Key Evolutionary Stages
    
% % %     1. Proto-Indo-European (PIE):
% % %         - PIE had a negation root ne- (not), which appears in many languages.
% % %         - Example: Latin "ne" (not), Greek "οὐ" (ou, not), Sanskrit "न" (na, not).
% % %     2. Old English (before ~1150 CE):
% % %         - The standard form was ne… ne (double negation).
% % %         - Example: "Ne he ne ic" ("Neither he nor I").
% % %         - Over time, one "ne" was replaced by "nor", which developed from a contraction of "ne" + "or" (meaning "not… or").
% % %     3. Middle English (~1150–1500 CE):
% % %         - The word "nor" became common as a conjunction in negative contexts.
% % %         - Example: "He wente not, nor wolde he speke."
% % %         - The first "ne" started disappearing, leading to the single "nor" we use today.
% % %     4. Modern English (~1500 CE–Present):
% % %         - The redundant "ne" was completely dropped in most cases.
% % %         - Today, "nor" is mainly used after "neither" or in formal negative clauses.
    

% % %         佢存在。 (*Keoi5 cyun4 zoi6.*)

% % % ➡️ "It exists."

% % % 2️⃣ 佢唔存在。 (*Keoi5 m4 cyun4 zoi6.*)

% % % ➡️ "It does not exist."

% % % 3️⃣ 佢又存在,又唔存在。 (*Keoi5 jau6 cyun4 zoi6, jau6 m4 cyun4 zoi6.*)

% % % ➡️ "It exists, but at the same time, it does not exist."

% % % 4️⃣ 佢冇得講。 (*Keoi5 mou5 dak1 gong2.*)

% % % ➡️ "It is inexpressible / indescribable."

% % % 5️⃣ 佢存在,但係又冇得講。 (*Keoi5 cyun4 zoi6, daan6 hai6 jau6 mou5 dak1 gong2.*)

% % % ➡️ "It exists, but at the same time, it is inexpressible."

% % % 6️⃣ 佢唔存在,但係又冇得講。 (*Keoi5 m4 cyun4 zoi6, daan6 hai6 jau6 mou5 dak1 gong2.*)

% % % ➡️ "It does not exist, but at the same time, it is inexpressible."

% % % 7️⃣ 佢存在,佢又唔存在,仲要冇得講。 (*Keoi5 cyun4 zoi6, keoi5 jau6 m4 cyun4 zoi6, zung6 jiu3 mou5 dak1 gong2.*)

% % % ➡️ "It exists, it does not exist, and yet it is still inexpressible."



% % % ### Thermippos — The Complete Dialogue

% % % *The scene is the agora, outside the office of the magistrate. Socrates is on his way to answer charges of impiety. There he meets Thermippos holding forth confidently amidst a gathering of young men. Naturally, since death is on his mind, Socrates seizes the opportunity to discuss the subject with a man who seems certain of everything.*

% % % Socrates. You agree, Thermippos, that all men are mortal.

% % % Thermippos. I do.

% % % Socrates. And you agree furthermore that I am a man.

% % % Thermippos. I have no reason to doubt it, Socrates.

% % % Socrates. Surely then you agree that I am mortal.

% % % Thermippos. I didn’t say that. You did. Don’t put words in my mouth.

% % % Socrates. I beg your pardon, Thermippos, but I have simply drawn what follows.

% % % Thermippos. Strawman.

% % % Socrates. But no true reasoner could fail —

% % % Thermippos. Ah, the no-true-Macedonian fallacy.

% % % Socrates. But, Thermippos, given the logical form . . .

% % % Thermippos. Define “logical form”.

% % % Socrates. . . . you must either accept the conclusion or reject at least one of the premises.

% % % Thermippos. False dichotomy.

% % % Socrates. I see, Thermippos. You’re an idiot.

% % % Thermippos. And that’s an ad hominem.

% % % *Socrates ad-hominems Thermippos with a brick. The charges of impiety are dropped.*

% % % 《泰米甫斯全對話》

% % % 場景:雅典市集,法院門口。蘇格拉底正準備去應付不敬神的指控,路過見到泰米甫斯喺度對住一班後生仔高談闊論。蘇格拉底心中惦記住死亡,見到有個乜都講得咁有信心嘅人,當然唔會放過傾下機會。

% % % ---

% % % 蘇格拉底: 你認同啦,所有人都會死。

% % % 泰米甫斯: 認同。

% % % 蘇格拉底: 咁你都認同,我係一個人。

% % % 泰米甫斯: 冇理由唔信你係人啦,蘇格拉底。

% % % 蘇格拉底: 咁你咪都要認同,我會死。

% % % 泰米甫斯: 我冇講過喎。你講咋。唔好擺啲嘢落我嘴度。

% % % 蘇格拉底: 唔好意思呀泰米甫斯,我淨係根據邏輯推論咋。

% % % 泰米甫斯: 稻草人詭辯。

% % % 蘇格拉底: 但係,凡係真心用邏輯思考嘅人,冇理由會——

% % % 泰米甫斯: 噢,「無真馬其頓人」謬誤嚟喇。

% % % 蘇格拉底: 但係泰米甫斯,按邏輯形式嚟講……

% % % 泰米甫斯: 你解釋下「邏輯形式」係乜先。

% % % 蘇格拉底: ……你要嘛接受結論,要嘛就否定其中一個前提。

% % % 泰米甫斯: 假二擇一。

% % % 蘇格拉底: 睇嚟你真係個戇居嘅人。

% % % 泰米甫斯: 人身攻擊!

% % % (蘇格拉底用磚頭人身攻擊咗泰米甫斯。)

% % % 結果:蘇格拉底不敬神嘅指控被撤銷。



% % % If the multitude of primes were finite, let them all be named. Multiply them together, and to this product, add one. This new number shares no measure with any named prime, for upon division it leaves one. Thus, either it is itself prime, or yields a new prime not among those named. Therefore, the primes are not bounded in number.

% % % 如果話世上嘅質數數晒得晒,就攞哂嚟數一數。

% % % 將佢哋全部乘埋一齊,再加一。

% % % 咁個新數,冇一個舊質數可以啱啱除得盡,因為每個都會剩返一。

% % % 所以呢個新數唔係舊質數之一,佢或者自己都係質數,

% % % 或者分解落去會出一隻新質數。

% % % 咁即係話,質數係數極都數唔晒㗎。



% % % https://abccanto.blogspot.com/2017/02/idioms-of-improbability.html

% % % 有一種修辭手法用黎誇張地形容一件事永遠冇可能發生,或者極微機會發生,英文叫

% % % adynaton: A [form](http://zh.m.wordow.com/english/dictionary/form) of [hyperbole](http://zh.m.wordow.com/english/dictionary/hyperbole) that [uses](http://zh.m.wordow.com/english/dictionary/uses) [exaggeration](http://zh.m.wordow.com/english/dictionary/exaggeration) so [magnified](http://zh.m.wordow.com/english/dictionary/magnified) as to [express](http://zh.m.wordow.com/english/dictionary/express) [impossibility](http://zh.m.wordow.com/english/dictionary/impossibility).

% % % (有冇人識中文點叫?)

% % % | Idiom | Literal Translation | Notes |
% % % | --- | --- | --- |
% % % | 猪乸會上樹 | "The female pig will climb a tree" | Freakishly similar to the English idiom "When pigs fly"!(Frequently used with a preceding [type of person] 靠得住...) |
% % % | 太陽由西邊出黎 | "The sun will rise from the West" | Also used in Mandarin |
% % % | 咁我係李嘉誠個仔 | (If [improbable thing is true], then “I am the son of Li Ka Shing!”) | Li Ka Shing is one of the richest persons in HK. Being his son (and thus being able to inherit his fortunes) is impossible |
% % % | 俾個港督你做 | “I’ll give you the position of Governor of Hong Kong!” | I don’t seem to have heard 俾個特首你做 since 1997. Maybe no one wants to be the Chief Executive of HKSAR? |
% % % | 批個頭落嚟畀你當凳坐 | “I’ll cut off my own head and let you sit on it” | When you are so confident betting against something that you would bet your head off, so to speak |
% % % | (如果… )我就唔姓X | “Then my surname won’t be X [X= your real surname]” | Surnames (last names) bear the family pride. You are so confident about the improbability of something that you would put your surname on the table. |

% % % 仲有冇其他大家聽過既表達方法?