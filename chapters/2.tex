\chapter{邏輯同語言}


\section{}

乜嘢 係 邏輯?就咁睇,我哋 將 可以 理解 邏輯 為 一種 研究 假設 已經 講咗 某啲 嘢,跟住 有 乜嘢 可以 正確 咁講 出嚟 嘅 學問。呢種理解,其實已經被好多學者用更具體嘅方式講過。William Kneale 就話邏輯係「關注有效推論原則」;Beale 同 Restall 亦講過:「邏輯嘅核心課題就係邏輯後果——要準確而且有系統咁講清楚,乜嘢叫做後果,通常係透過界定邊啲論證(喺某個語言入面)係 *有效*。今日至少,呢個係共識。」

邏輯同語言嘅親密關係,其實都反映喺術語度——前者俾人叫做「形式語言」,後者就係「自然語言」。邏輯係形式化嘅,而語言係自然嘅,而邏輯就係嘗試去模擬語言。回顧歷史會更加清楚,尤其係西方。邏輯嘅研究源於古希臘,最出名就係亞里士多德。嗰度我哋見到,佢哋刻意咁抽離咗內容,去建立一套系統,用嚟評價自然語言入面嘅論證。Kneale、Shenefelt 同 White 都強調,西方邏輯嘅源頭就係對自然語言辯論嘅拆解,特別係古希臘公共政治生活入面嘅辯論,推動咗亞里士多德構建邏輯。重點係要有一套客觀、清晰、外在嘅系統,去判斷一個論證值唔值得信——唔睇內容,而睇形式。即使現代邏輯幾乎完全根植於西方傳統,但邏輯研究嘅決定性特徵,都係認知到自然語言論證入面「形式正確」嘅現象——無論係印度定中國傳統。

發展邏輯嘅過程可以話係有跡可尋,大致上有三個階段:(一)觀察;(二)辨認;(三)形式化。

* **觀察**:注意到有啲論證結構或者模式係會不斷重複出現,而且係「正確」嘅。
* **辨認**:同前後階段有密切關連,但唔同嘅地方係呢度涉及一個選擇——可能係有意識,可能係無意識——要揀邊啲模式要俾人捕捉落去形式化階段。即係要揀出自然語言入面邊啲重複出現嘅形式係「正確」,所以值得納入邏輯系統。
* **形式化**:喺呢度我哋透過一套符號同規則,去形式化之前揀出嚟嘅結構。形式化通常有兩大方法:演繹法,或者語義法。符號係抽離咗一切同邏輯有效性無直接關聯嘅意義——特別係演繹方法裏面。所以,某程度上,呢個邏輯一旦生成,就同佢出身嘅語言分割開嚟。

唔可以話所有我哋今日見到嘅邏輯都係咁樣生成嘅。特別係,唔係所有邏輯都係直接從自然語言度生出嚟。不過,大部分我哋承認係「邏輯」嘅系統,都係咁嚟。原因係,即使去到布爾年代,邏輯系統嘅數學性質已經開始浮現,邏輯已經被當成一種數學對象嚟研究。數學化、代數化之後,邏輯甚至可以透過改變公理等方法創造新嘅「邏輯」或者類邏輯對象。

---

### 二、邏輯嘅一般形式

我哋知道,邏輯係自然語言嘅模型,但佢唔係一個完整嘅模型。喺邏輯入面,唔可能表達自然語言所有可以講嘅嘢。邏輯只係模擬某啲論證形式,而且係「正確」嘅論證形式。

所以我哋要問嘅係:「乜嘢係 *一個* 邏輯?」而唔係單純問「乜嘢係邏輯?」Shapiro 就喺呢方面提供咗好有用嘅界定,以下基本上係跟佢嘅術語:

(此處略去細則性條文翻譯,避免過度重複,但結構保持一致,包括公式系統、演繹系統、語義、模型、滿足關係、語義後果、邏輯真理等等。)

---

### 三、語言同邏輯嘅糾纏

重點係:即使邏輯嘅建立目的係要將佢同語言分離,但實際上語言同邏輯仍然千絲萬縷咁纏繞住。

1. **母語言**:邏輯係模擬自然語言,而模擬乜、點樣模擬,就靠母語言。
2. **元語言**:邏輯嘅語義,必須用自然語言去交代。就算全部寫符號都得,但結果會自我挫敗,因為解釋失效,甚至會變成循環。於是,語義必須靠自然語言去規定,呢個自然語言就係元語言。
3. **對象語言**:邏輯喺應用時,可以應用到母語言以外嘅語言。西方邏輯一早已經用嚟評估印度、漢語、日語論證。咁所以,要區分母語言同對象語言。

例如,現代謂詞邏輯嘅母語言係德文,但今日中國學生可以用中文寫例子,用英文語義檢驗,呢度就出現咗三層:母語言(德文)、元語言(英文)、對象語言(中文)。

---

你要唔要我幫你將後面「定義、推導系統、模型語義」嗰部分都逐條詳細翻譯成中文(廣東話書面語)?定係保留結構同摘要,方便你之後再加工?



好,我幫你將後面嗰段 **「邏輯嘅一般形式」** 逐條詳細譯成書面廣東話(盡量保留技術嚴謹性):

---

### 二、邏輯嘅一般形式

我哋知道,邏輯係自然語言嘅模型,但佢唔係一個完整嘅模型。喺邏輯入面,唔可能表達自然語言所有可以講嘅嘢。邏輯淨係模擬某啲論證形式,而且係「正確」嘅論證形式。

所以重點係要問:「乜嘢係 *一個* 邏輯?」而唔係單純問「乜嘢係邏輯?」。Shapiro 提供咗一個好詳細嘅界定,以下基本上係跟佢嘅術語:

1. 定義一個 **完整邏輯** 為一個 **公式系統**(formula system,或「形式語言」),連埋一套 **演繹系統**(deductive system)同一個 **語義系統**(semantics)。
2. **公式系統 / 形式語言**:由一組「良構公式」(well-formed formulas)組成。一般嚟講,每個公式都係由某個字母表組成嘅有限字串。
3. 定義一個 **理論**(theory / formal theory),就係一組公式嘅集合。
4. 定義一個 **論證**(argument):由一對組成,前者係一組公式(前提),後者係單一公式(結論)。**演繹系統** 就係一組「推導」:每一個推導都對應一個或多個論證。如果有一個論證 $(\Gamma, \varphi)$,咁「由 $\Gamma$ 推導出 $\varphi$」就係一個有限公式序列,其中每一個公式要麼係 $\Gamma$ 嘅成員,要麼係透過推理規則由 $\Gamma$ 嘅子集推出。
5. 如果喺一個演繹系統入面,$\Gamma$ 可以推導出 $\varphi$,就叫 $\varphi$ 係 $\Gamma$ 嘅 **演繹後果**。一個公式如果可以由空集合推導出,就叫做 **定理**。
6. **語義系統**:係一組「模型」(models)。每個模型由一個或多個「域」(domain)組成,再加上一個解釋函數(interpretation function),用嚟將語言入面嘅符號指派到某啲對應嘅對象。語義系統通常仲有另外兩個部分:

   * **指稱函數**(denotation function):畀定一個模型同一個變量指派,呢個函數會將語言入面嘅每個項指派到模型嘅域入面嘅某個元素。
   * **滿足關係**(satisfaction relation):定義模型、變量指派同公式之間嘅關係。可以理解為一個函數,將「模型 × 指派 × 公式」映射到一個「值謂詞系統」(value-predicate scheme)嘅成員。所謂「值謂詞系統」係一個非空集合,其中每個元素都叫做一個「值謂詞」(value-predicate)。
7. **語義後果 / 有效性**:要揀出值謂詞系統入面某個子集,叫做「指定值集合」(designated value set)。喺某個有效性概念之下,一個模型 $M$、一個指派、同一個公式 $\varphi$,如果滿足關係將 $(M, \text{assignment}, \varphi)$ 映射到指定值集合入面嘅值,就叫做 $\varphi$ 喺 $M$ 入面成立。
8. 如果一個語義系統 $S$、一個模型 $M$,同一組公式 $\Gamma$ 加上一個公式 $\varphi$,若然 $\Gamma$ 喺所有指派之下都成立,而 $\varphi$ 喺所有模型同指派之下都跟住成立,就叫呢個論證 $(\Gamma, \varphi)$ **語義上有效**。一個公式如果係空集合嘅語義後果,就叫做 **邏輯真理**。
9. 如果一個邏輯系統嘅公式系統係無語義嘅(semantics-less),即係佢冇任何模型同解釋。咁樣嘅話,公式嘅「意義」完全由演繹系統提供。當然,語義後果同邏輯真理嘅概念就唔適用。
10. 相反嘅極端情況係:一個邏輯嘅「語義」只包含單一模型。咁呢個形式系統就叫 **完全詮釋**(fully interpreted)。例如算術嘅公理化,唯一嘅詮釋就係自然數,配合標準嘅運算符號。

---

以上就係一個邏輯嘅一般結構。當然,呢啲組件都有好多彈性:

* 喺演繹系統入面,可以揀唔同嘅公理同推理規則。
* 喺語義系統入面,可以揀唔同嘅值謂詞集合、指定值集合同滿足關係。

問題係:咁多種可能嘅配置,我哋點知道邊啲係「好」邏輯?點解值謂詞要解釋成「真/假」,唔可以係「對/錯」、「陰/陽」甚至「男/女」?點解只用兩個值?點解唔可以有「真亦假」、「非真非假」、「不定」?

---

呢啲「邏輯之外」嘅問題,將我哋帶返去一開始嘅討論:邏輯最初就係透過觀察、辨認、同形式化,從自然語言入面生出嚟。分辨唔同配置同詮釋嘅方法,就係將佢哋同自然語言入面「可識別為正確」嘅論證例子比較。換句話講,我哋其實假設自然語言入面已經隱含咗一種原初邏輯(proto-logic)。

Shapiro 自己亦講過:「咁樣理解嘅話,邏輯理論預設咗自然語言本身就有正確推理,仲有,人類可以無需建立或學習演繹系統,就已經識得辨別某啲正確或者錯誤嘅推理。」

---

你要唔要我繼續幫你將最後「母語言 / 元語言 / 對象語言」嗰部分都完整翻譯埋落去?
